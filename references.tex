\section{References}
References can be listed in any standard referencing style that uses a {\bf numbering} system (i.e. not Harvard referencing style), and should be consistent between references within a given article. However, key points include:
\begin{itemize}
\item Journal abbreviations should follow the Index Medicus/MEDLINE abbreviation approach.
\end{itemize}
\begin{itemize}
\item Datasets should be cited in the reference section and should follow one of these examples.
\end{itemize}
\begin{itemize}
\item Only articles, datasets and abstracts that have been published or are in press, or are available through public e-print/preprint servers/data repositories, may be cited. Unpublished abstracts, papers that have been submitted but not yet accepted, and personal communications should instead be included in the text, and should be referred to as ‘personal communications’ or ‘unpublished reports’ and the researchers involved should be named. It is the responsibility of the authors to ensure they obtain permission to quote any personal communications from the cited individuals.
\end{itemize}
\begin{itemize}
\item Web links, URLs, and links to the authors’ own websites should be included as hyperlinks within the authors' manuscript (e.g. 'Mouse Tumor Biology Database'), and not as references.
\end{itemize}
\begin{itemize}
\item References to trials on a clinical trial database should be as follows:
[Authors/name of group], [title of the trial], In: ClinicalTrials.gov [cited year month date], Available from [URL of the link from ClinicalTrials.gov] e.g. Kovacs Foundation, The Effect of Ozone Therapy for Lumbar Herniated Disc. In: ClinicalTrials.gov [cited 2012 Aug 30], Available from http://clinicaltrials.gov/ct2/show/NCT00566007
\end{itemize}
